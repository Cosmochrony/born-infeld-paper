\begin{abstract}
  We investigate the emergence of non-linear electrodynamics and spacetime geometry from
  relational systems whose effective continuum descriptions admit a finite maximal propagation or relaxation flux.
  We show that bounded propagation excludes purely
  quadratic effective actions and uniquely enforces a Born--Infeld--type structure as the
  minimal local representation compatible with flux saturation.
  Starting from a weighted relational Laplacian endowed with an irreversible relaxation dynamics, we derive an
  effective action of the form
  \(\sqrt{-\det(g_{\mu\nu}+\alpha F_{\mu\nu})}\),
  where the gauge field arises as an antisymmetric perturbation of relational connectivity.
  In homogeneous regimes, the resulting operator selects flat spacetime with pseudo-Riemannian signature \((- + + +)\).
  In the presence of a localized and stationary obstruction, the same mechanism yields the Schwarzschild geometry as the
  universal effective exterior solution.
  We further show that horizons correspond to saturation of the bounded-flux condition and to a loss of projectability
  of the effective description, rather than to physical singularities.
  These results provide a field-theoretic and operator-based derivation of Born--Infeld electrodynamics and its
  associated geometries from bounded relational relaxation.
\end{abstract}
