\section{Continuum limit and Born--Infeld action}
  \label{sec:continuum-limit-and-effective-operators}

  We briefly recall how an effective continuum description and its associated action
  emerge from the spectral structure of a bounded relational Laplacian.
  Detailed proofs of the continuum limit and operator convergence can be found
  in~\cite{BeauRelationalSpectral}; here we summarize only the elements required to
  establish the effective field-theoretic description.

  We consider a sequence of weighted graphs $G_N=(V_N,E_N,w^{(N)})$ with increasing
  cardinality $|V_N|\to\infty$.
  The graphs are assumed to become dense in the sense that each node has an increasing
  number of neighbors, while the weights $w_{ij}^{(N)}$ decay sufficiently fast with an
  intrinsic relational distance.
  No background embedding is assumed; all notions of proximity are defined intrinsically
  from the graph structure.

  Let $\phi_i$ denote a scalar degree of freedom on the nodes of $G_N$.
  The discrete Laplacian acts as
  \begin{equation}
    (L_N \phi)_i = \sum_j w_{ij}^{(N)}\left(\phi_i-\phi_j\right).
  \end{equation}
  Under mild regularity, isotropy, and density assumptions, the action of $L_N$ on slowly
  varying configurations converges to that of a second-order differential operator on a
  smooth manifold $\mathcal{M}$,
  \begin{equation}
    L_N \;\longrightarrow\;
    \mathcal{L}
    = \nabla_\mu\!\left(A^{\mu\nu}(x)\nabla_\nu\right),
  \end{equation}
  where $A^{\mu\nu}(x)$ is a symmetric tensor encoding the local connectivity and
  stiffness of the underlying relational system.

  In admissible continuum regimes, the operator $\mathcal{L}$ is elliptic.
  Its principal symbol,
  \begin{equation}
    \sigma_2(\mathcal{L})(x,k)=A^{\mu\nu}(x)k_\mu k_\nu,
  \end{equation}
  fully characterizes the leading-order propagation of modes.
  As shown in~\cite{BeauRelationalSpectral}, this symbol provides a natural,
  coordinate-independent definition of an effective metric tensor,
  \begin{equation}
    g^{\mu\nu}(x)\propto A^{\mu\nu}(x),
  \end{equation}
  where the proportionality factor reflects a choice of units.

  Crucially, the effective metric $g_{\mu\nu}$ is not introduced as an independent
  dynamical variable.
  It is a derived object, encoding how relational variations propagate in the continuum
  limit.
  Different microscopic graphs leading to the same operator $\mathcal{L}$ are therefore
  indistinguishable at the level of effective geometry.

  When antisymmetric perturbations of the relational connectivity are included, the
  continuum operator acquires an additional two-form contribution that enters naturally
  as a field strength $F_{\mu\nu}$.
  As discussed in Section~\ref{sec:bounded-relaxation-and-effective-action}, bounded relaxation constrains the
  admissible gradients of these perturbations.
  At the level of the effective continuum description, this constraint is most naturally
  implemented at the level of the action rather than the equations of motion.

  The resulting effective action takes a Born--Infeld form,
  \begin{equation}
    S_{\mathrm{eff}}
    = \beta^2 \int d^4x
    \left(
      \sqrt{-\det\!\left(g_{\mu\nu}+\frac{1}{\beta}F_{\mu\nu}\right)}
      -\sqrt{-\det(g_{\mu\nu})}
    \right),
    \label{eq:BI-action-continuum}
  \end{equation}
  where $\beta$ sets the maximal admissible relational flux.
  In the weak-field regime, this action reduces to the standard Maxwell form, while at
  large field strengths it enforces saturation of the effective dynamics.

  The validity of this effective action is restricted to regimes in which the spectrum of
  $L_N$ admits a well-defined low-energy sector and the bounded-flux condition holds.
  Outside these regimes, the continuum operator $\mathcal{L}$ ceases to provide an
  adequate description, and geometric or field-theoretic notions lose their operational
  meaning.

  In the following section, we show that the bounded-relaxation constraint further
  restricts the admissible forms of $A^{\mu\nu}(x)$.
  In homogeneous regimes, these restrictions uniquely select a flat effective geometry
  with a specific signature.
