\section{Continuum limit and effective operators}
  \label{sec:continuum-limit-and-effective-operators}

  We briefly recall how an effective continuum description emerges from the spectral
  structure of a relational Laplacian.
  Detailed proofs and technical constructions can be found in~\cite{BeauRelationalSpectral}; here we
  summarize only the elements required for the subsequent analysis.

  We consider a sequence of weighted graphs $G_N=(V_N,E_N,w^{(N)})$ with increasing
  cardinality $|V_N| \to \infty$.
  We assume that the graphs become dense in the sense that each node has an increasing
  number of neighbors, while the weights $w_{ij}^{(N)}$ decay sufficiently fast with a
  relational distance scale.
  No background embedding is assumed; all notions of proximity are defined intrinsically
  from the graph structure.

  Let $\phi_i$ denote a scalar field on the nodes of $G_N$.
  The discrete Laplacian acts as
  \begin{equation}
    (L_N \phi)_i = \sum_j w_{ij}^{(N)} \left( \phi_i - \phi_j \right) .
  \end{equation}
  Under mild regularity, isotropy, and density assumptions, the action of $L_N$ on
  slowly varying configurations converges to that of a second-order differential
  operator on a smooth manifold $\mathcal{M}$,
  \begin{equation}
    L_N \;\longrightarrow\;
    \mathcal{L}
    = \nabla_\mu \!\left( A^{\mu\nu}(x)\,\nabla_\nu \right) ,
  \end{equation}
  where $A^{\mu\nu}(x)$ is a symmetric, positive-definite tensor field encoding the local
  connectivity structure of the underlying relational system.

  The operator $\mathcal{L}$ is elliptic in regimes where a continuum description is
  admissible.
  Its principal symbol,
  \begin{equation}
    \sigma_2(\mathcal{L})(x,k) = A^{\mu\nu}(x)\,k_\mu k_\nu ,
  \end{equation}
  fully characterizes the leading-order propagation of modes.
  As shown in~\cite{BeauRelationalSpectral}, this symbol provides a natural and coordinate-independent
  definition of an effective metric tensor,
  \begin{equation}
    g^{\mu\nu}(x) \propto A^{\mu\nu}(x) .
  \end{equation}
  The proportionality factor reflects a choice of units and plays no role in the
  following.

  Importantly, the metric $g_{\mu\nu}$ is not introduced as an independent dynamical
  variable.
  It is a derived object, encoding how relational variations propagate in the continuum
  limit.
  Different microscopic graphs leading to the same operator $\mathcal{L}$ are therefore
  indistinguishable at the level of effective geometry.

  The validity of this geometric description is restricted to regimes in which the
  spectrum of $L_N$ admits a well-defined low-energy sector and where the bounded-flux
  condition discussed in Section~2 is satisfied.
  Outside these regimes, the continuum operator $\mathcal{L}$ ceases to provide an
  adequate description, and geometric notions lose their operational meaning.

  In the following section, we show that the bounded-flux constraint imposes strong
  restrictions on the admissible forms of $A^{\mu\nu}(x)$.
  In homogeneous relaxation regimes, these restrictions uniquely select a flat
  effective geometry with a specific signature.
