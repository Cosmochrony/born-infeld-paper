\section{Localized obstruction and Schwarzschild geometry}
  \label{sec:localized-obstruction-and-schwarzschild-geometry}

  We now consider relaxation regimes in which homogeneity is broken by the presence of a
  localized and stationary obstruction.
  Operationally, such an obstruction corresponds to a region where the relational
  connectivity or stiffness of the underlying system is enhanced, thereby constraining
  the local relaxation dynamics.
  No assumption is made regarding the microscopic origin of this obstruction; only its
  macroscopic effect on the effective operator is considered.

  Outside the obstructed region, the system remains stationary and isotropic.
  The effective continuum operator introduced in Section~4 therefore takes the form
  \begin{equation}
    \mathcal{L}
    = \nabla_\mu \!\left( A^{\mu\nu}(r)\,\nabla_\nu \right) ,
  \end{equation}
  where $r$ denotes the radial distance from the center of the obstruction.
  Spherical symmetry implies that $A^{\mu\nu}(r)$ depends only on $r$ and decomposes into
  temporal and spatial components,
  \begin{equation}
    A^{\mu\nu}(r) =
    \mathrm{diag}\!\left(-A_t(r),\,A_r(r),\,A_\perp(r),\,A_\perp(r)\right) .
  \end{equation}

  In the exterior region, where no sources are present, the relaxation dynamics is governed
  by the homogeneous equation
  \begin{equation}
    \nabla_\mu \!\left( A^{\mu\nu}(r)\,\nabla_\nu \Phi \right) = 0 ,
  \end{equation}
  where $\Phi$ represents a slowly varying scalar probe of the effective structure.
  Under stationarity and spherical symmetry, this equation reduces to
  \begin{equation}
    \frac{1}{r^{2}}
    \frac{d}{dr}
    \left(
      r^{2} A_r(r) \frac{d\Phi}{dr}
    \right)
    = 0 .
  \end{equation}
  Its general solution is
  \begin{equation}
    \Phi(r) = \Phi_0 - \frac{C}{r} ,
  \end{equation}
  where $C$ is an integration constant characterizing the strength of the obstruction.

  The emergence of a $1/r$ profile is therefore not imposed but follows generically from
  flux conservation in a stationary and isotropic relaxation regime.
  This behavior is independent of the detailed microscopic realization of the
  obstruction and reflects the universal structure of the effective operator.

  Using the identification between the principal symbol of $\mathcal{L}$ and the effective
  metric established in Section~4, the radial dependence of $A^{\mu\nu}(r)$ translates
  directly into a position-dependent metric tensor.
  Up to a choice of coordinates, the effective line element can be written as
  \begin{equation}
    ds^{2}
    = - f(r)\, dt^{2}
    + f(r)^{-1} dr^{2}
    + r^{2} d\Omega^{2} ,
  \end{equation}
  where the lapse function $f(r)$ is determined by the radial profile of the operator
  coefficients.
  Matching the weak-obstruction limit with the homogeneous solution of
  Section~5 fixes
  \begin{equation}
    f(r) = 1 - \frac{r_s}{r} ,
  \end{equation}
  where $r_s$ is a constant proportional to $C$.

  The resulting geometry coincides with the Schwarzschild metric.
  Here, however, it arises as an effective description of a stationary relaxation pattern
  around a localized obstruction, rather than as a solution of an independent set of
  gravitational field equations.
  The parameter $r_s$ characterizes the strength of the obstruction in relational units
  and is identified operationally through its influence on propagation and relaxation
  rates.

  This derivation highlights that the Schwarzschild geometry is a universal effective
  response to a localized and isotropic perturbation in a bounded relaxation framework.
  No additional assumptions regarding curvature dynamics or energy--momentum sources are
  required.
  The geometry encodes how admissible modes propagate in the presence of constrained
  relaxation, and its form is fixed by symmetry, stationarity, and flux conservation alone.

  In the next section, we analyze the behavior of the effective operator near the radius
  $r = r_s$.
  We show that this surface corresponds to a saturation of the bounded-flux condition,
  leading to a loss of projectability and providing an operator-theoretic interpretation
  of horizons.
