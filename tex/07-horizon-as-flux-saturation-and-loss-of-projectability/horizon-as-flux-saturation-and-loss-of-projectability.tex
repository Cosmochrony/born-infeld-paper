\section{Horizon as flux saturation and loss of projectability}
  \label{sec:horizon-as-flux-saturation-and-loss-of-projectability}

  We now examine the behavior of the effective continuum description in the vicinity of
  the radius $r=r_s$ identified in Section~6.
  At this radius, the lapse function $f(r)$ vanishes and the effective geometry develops a
  horizon in the standard geometric representation.
  We show that, in the present framework, this phenomenon admits a precise and purely
  operator-theoretic interpretation in terms of flux saturation.

  As established in Section~2, the effective action and the associated continuum operator
  remain valid only as long as the bounded-flux condition is satisfied.
  In the presence of a localized obstruction, the radial dependence of the operator
  coefficients $A^{\mu\nu}(r)$ reflects the increasing constraint imposed on relational
  relaxation.
  As $r$ decreases, admissible gradients approach the maximal value set by the saturation
  scale $\beta$.
  At $r=r_s$, this bound is reached.

  Beyond this point, the effective operator can no longer sustain propagating modes
  compatible with bounded relaxation.
  While the differential expression for $\mathcal{L}$ remains formally defined, its
  principal symbol becomes degenerate at $r=r_s$.
  As a result, the reconstruction of a local effective metric from the operator symbol
  ceases to be well-defined.
  This breakdown signals the loss of validity of the geometric description rather than
  the presence of a physical singularity.

  From the perspective of the effective action, the horizon corresponds to the locus
  where the Born--Infeld saturation becomes operative.
  At this point, the non-linear structure of the action enforces a maximal admissible
  field strength, preventing divergences in energy density and flux.
  As in standard Born--Infeld electrodynamics, the saturation mechanism ensures the
  finiteness of physically relevant quantities and excludes point-like singular sources.

  From the viewpoint of the underlying relational dynamics, the horizon separates two
  qualitatively distinct regimes.
  Outside $r_s$, the relaxation dynamics admits a faithful projection into a continuum
  description with well-defined geometric and field-theoretic observables.
  At and inside $r_s$, distinct microscopic configurations of the relational system become
  indistinguishable at the level of effective operators.
  The projection from relational states to effective spacetime observables is therefore
  non-injective in this regime.

  This interpretation is consistent with structural analyses of non-injective effective
  descriptions, in which the loss of observable factorization reflects a limitation of the
  effective description rather than a breakdown of the underlying dynamics~\cite{BeauNonInjectiveBell}.
  In the present context, the horizon is precisely the locus where such non-injectivity
  becomes unavoidable as a consequence of bounded flux propagation.

  Importantly, no extension of the effective spacetime geometry beyond the horizon is
  required for internal consistency.
  The continuum description is explicitly restricted to projectable regimes, and the
  horizon marks the boundary of its domain of applicability.
  Questions concerning the behavior of the relational system beyond this boundary are
  meaningful only at the level of the underlying dynamics and need not admit a geometric
  or field-theoretic representation.

  This operator-based interpretation reframes horizons as kinematic consequences of
  bounded relaxation rather than as geometric pathologies.
  It explains both the universality of horizon formation and the robustness of
  Schwarzschild geometry under variations of the microscopic relational structure, while
  clarifying why attempts to probe beyond the horizon using effective spacetime observables
  are intrinsically limited.
