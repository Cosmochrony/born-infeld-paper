\section{Horizon as flux saturation and loss of projectability}
  \label{sec:horizon-as-flux-saturation-and-loss-of-projectability}

  We now examine the behavior of the effective continuum operator in the vicinity of the
  radius $r=r_s$ identified in Section~6.
  At this radius, the lapse function $f(r)$ vanishes and the effective geometry develops
  a horizon in the usual geometric description.
  Here we show that this feature admits a precise operator-theoretic interpretation in
  terms of flux saturation and loss of projectability.

  As established in Section~2, the effective continuum description is valid only as long
  as the bounded-flux condition is satisfied.
  In the presence of a localized obstruction, the radial profile of the operator
  coefficients $A^{\mu\nu}(r)$ increases as $r$ decreases, reflecting the growing
  constraint imposed on relaxation.
  At $r=r_s$, the maximal admissible flux is reached.
  Beyond this point, the operator can no longer sustain propagating modes compatible with
  the bounded-flux condition.

  Operationally, this manifests as a degeneration of the effective operator.
  While the differential expression for $\mathcal{L}$ remains formally defined, its
  principal symbol ceases to be invertible at $r=r_s$.
  As a consequence, the reconstruction of a local effective metric from the operator
  symbol breaks down.
  This signals the failure of the geometric description rather than the appearance of a
  physical singularity.

  From the perspective of the relational dynamics, the horizon corresponds to a boundary
  between two regimes.
  Outside $r_s$, the relaxation dynamics admits a faithful projection into a continuum
  description with well-defined observables.
  At and inside $r_s$, distinct microscopic configurations of the relational system
  become indistinguishable at the level of effective operators.
  The projection from relational states to effective geometric observables is therefore
  non-injective in this regime.

  This interpretation aligns with structural analyses of non-injective projections, in
  which the loss of observable factorization marks a fundamental limitation of effective
  descriptions rather than a breakdown of the underlying dynamics~\cite{BeauNonInjectiveBell}.
  In the present context, the horizon is precisely the locus where such non-injectivity
  becomes unavoidable due to flux saturation.

  Importantly, no extension of the effective geometry beyond the horizon is required for
  the internal consistency of the description.
  The effective continuum framework is explicitly restricted to projectable regimes,
  and the horizon marks the boundary of its domain of applicability.
  Questions concerning the behavior of the relational system beyond this boundary are
  meaningful only at the level of the underlying dynamics and need not admit a geometric
  representation.

  This operator-theoretic interpretation reframes horizons as kinematic features of
  bounded relaxation rather than as geometric pathologies.
  It explains why horizon formation is universal and robust under variations of the
  microscopic structure, while simultaneously clarifying why attempts to probe beyond
  the horizon using effective spacetime observables are intrinsically limited.

  With this interpretation, the emergence of flat spacetime, Schwarzschild geometry, and
  horizon formation form a coherent hierarchy of effective descriptions selected by
  bounded relaxation dynamics.
  In the following discussion, we summarize the scope and limitations of this framework
  and outline directions for further investigation.
