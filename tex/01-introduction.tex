\section{Introduction}
  \label{sec:introduction}

  A growing body of work explores the possibility that spacetime geometry is not a
  fundamental structure but an effective description emerging from more primitive
  relational or operator-based systems.
  In several approaches, discrete relational data, weighted graphs, or spectral operators
  are shown to admit continuum limits in which geometric notions such as distance,
  dimension, and curvature arise as derived quantities rather than postulated ones.
  In particular, it has been demonstrated that, under mild regularity and density
  assumptions, the continuum limit of a relational Laplacian converges to a second-order
  elliptic operator whose principal symbol defines an effective metric tensor~\cite{BeauRelationalSpectral}.

  While such results establish that effective spacetime geometry can emerge from
  non-geometric relational structures, they leave open a central question:
  why are only very specific geometries observed?
  From a purely mathematical perspective, a large class of elliptic operators and
  corresponding metrics are admissible.
  However, not all such operators necessarily correspond to physically meaningful or
  operationally accessible descriptions.
  Recent analyses have emphasized that effective geometric descriptions may fail when
  the underlying relational structure admits non-injective or degenerate projections,
  leading to a loss of observable factorization and to intrinsic limits of spacetime
  representability~\cite{BeauNonInjectiveBell}.

  This work addresses a complementary and more restrictive problem.
  Rather than asking how geometry can emerge, we ask which effective geometries are
  \emph{dynamically admissible} once minimal physical constraints are imposed on the
  underlying relational dynamics.
  Specifically, we consider relational systems whose effective continuum descriptions
  admit a finite maximal propagation or relaxation speed.
  Such a bound is required for causal consistency and excludes purely quadratic actions,
  which allow arbitrarily large gradients and unbounded fluxes.

  We show that imposing bounded flux propagation uniquely constrains the form of any
  local effective functional governing the continuum limit.
  Under mild assumptions of locality, smoothness, and absence of additional microscopic
  scales, the resulting effective description must take a Born--Infeld--type form.
  This structure is not introduced as a phenomenological modification, but arises as the
  minimal representation compatible with flux saturation.

  Building on this result, we demonstrate that flux saturation does more than regularize
  the effective theory.
  It dynamically selects a restricted class of admissible relational Laplacians whose
  continuum limits define stable and physically meaningful metrics.
  In homogeneous regimes, this selection leads uniquely to flat spacetime.
  In the presence of a localized and stationary obstruction, the same mechanism yields
  the Schwarzschild geometry as the universal effective solution, without postulating
  any independent metric dynamics or field equations.

  Within this framework, horizons are interpreted as loci where flux saturation renders
  the effective operator degenerate, signaling a loss of projectability rather than a
  physical singularity.
  This interpretation aligns naturally with structural analyses of non-injective
  projections and clarifies the operational meaning of strong-field regimes.

  The paper is organized as follows.
  In Section~\ref{sec:bounded-relaxation-and-effective-action}, we establish the necessity of a Born--Infeld--type
  structure from bounded flux propagation.
  Section~\ref{sec:rel-graph-laplacian-op-and-relaxation-dynamics} recalls the emergence of effective
  continuum operators from relational Laplacians.
  Section~\ref{sec:continuum-limit-and-effective-operators} discusses metric reconstruction from operator symbols.
  Sections~\ref{sec:emergence-of-minkowski-spacetime} and~\ref{sec:localized-obstruction-and-schwarzschild-geometry}
  derive flat and Schwarzschild geometries as dynamically selected solutions.
  Section~\ref{sec:horizon-as-flux-saturation-and-loss-of-projectability} analyzes horizon formation as operator
  saturation.
  We conclude with a discussion of scope and limitations.
