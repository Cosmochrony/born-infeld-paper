\section{Discussion and conclusion}
  \label{sec:discussion-and-conclusion}

  In this work, we have investigated the emergence and selection of effective spacetime
  geometries from a relational dynamics subject to bounded flux propagation.
  Starting from a weighted relational graph endowed with an irreversible relaxation
  process, we have shown that minimal and physically motivated constraints suffice to
  drastically restrict the class of admissible continuum descriptions.

  A first central result is that bounded propagation excludes purely quadratic effective
  actions.
  Requiring locality, smoothness, and the absence of additional microscopic scales leads
  uniquely to a Born--Infeld--type structure as the minimal effective representation
  compatible with flux saturation.
  This structure is not postulated as a modification of known dynamics, but arises as a
  necessary condition for causal consistency in any continuum limit admitting a maximal
  propagation speed.

  Building on established results concerning the continuum limit of relational
  Laplacians~\cite{BeauRelationalSpectral}, we have shown that the bounded-flux condition does more than regularize
  the effective theory.
  It dynamically selects the form of admissible operators whose principal symbols define
  effective metrics.
  In homogeneous and isotropic relaxation regimes, this selection uniquely yields a flat
  spacetime with pseudo-Riemannian signature $(- + + +)$.
  Minkowski geometry thus emerges as a dynamically admissible fixed point rather than as a
  fundamental background.

  When homogeneity is broken by a localized and stationary obstruction, the same framework
  leads generically to a $1/r$ relaxation profile.
  Through the operator--metric correspondence, this profile induces the Schwarzschild
  geometry as the universal effective description of the exterior region.
  Importantly, this result follows from symmetry, stationarity, and flux conservation
  alone, without invoking independent metric dynamics or gravitational field equations.

  We have further shown that horizons admit a natural operator-theoretic interpretation.
  They correspond to loci where the bounded-flux condition is saturated and where the
  principal symbol of the effective operator becomes degenerate.
  At this point, the reconstruction of a local geometric description fails, signaling a
  loss of projectability rather than the presence of a physical singularity.
  This interpretation is consistent with structural analyses of non-injective projections
  and clarifies the operational meaning of strong-field regimes~\cite{BeauNonInjectiveBell}.

  The scope of the present work is deliberately limited.
  We do not propose a complete theory of gravitation, nor do we claim to describe the
  microscopic origin of the relational dynamics.
  Our results apply only in regimes where a continuum description is meaningful and where
  bounded relaxation holds.
  Outside these regimes, geometric notions are not expected to remain valid, and no
  extension of the effective spacetime description is implied.

  Within these limits, the framework provides a unified and minimal explanation for the
  emergence of flat spacetime, Schwarzschild geometry, and horizons as dynamically selected
  effective structures.
  It suggests that key features traditionally attributed to gravitational dynamics may
  instead reflect universal properties of bounded relational relaxation.
  Further work will be required to explore non-stationary regimes, departures from
  spherical symmetry, and possible observational or phenomenological consequences of this
  operator-based perspective.
