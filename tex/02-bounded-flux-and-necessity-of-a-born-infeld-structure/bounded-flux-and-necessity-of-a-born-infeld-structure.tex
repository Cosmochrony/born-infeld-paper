\section{Bounded flux and effective action}
  \label{sec:bounded-flux-and-necessity-of-a-born--infeld-structure}

  We consider effective continuum descriptions arising from relational or operator-based
  systems whose microscopic dynamics admits a finite maximal propagation or relaxation
  speed.
  Such a bound is a minimal physical requirement for causal consistency, ensuring that
  no influence, signal, or constraint can propagate arbitrarily fast in the effective
  description.

  Let $\phi$ denote a generic effective field encoding the coarse-grained degrees of
  freedom of the relational system.
  We assume that, in regimes where a continuum description is meaningful, the effective
  dynamics can be represented by a local action functional of the form
  \begin{equation}
    S = \int \mathcal{L}(\partial_\mu \phi)\, d^n x ,
  \end{equation}
  where the Lagrangian density depends only on first derivatives of $\phi$.
  No background geometry is assumed at this stage; spacetime notions are introduced only
  as effective descriptive tools.

  A purely quadratic functional,
  \begin{equation}
    \mathcal{L}_{\mathrm{quad}} = \frac{1}{2}\,\partial_\mu \phi\,\partial^\mu \phi ,
  \end{equation}
  fails to enforce any upper bound on the magnitude of the gradient $\partial_\mu \phi$.
  As a result, the associated fluxes and characteristic propagation speeds are unbounded.
  This allows arbitrarily sharp gradients and instantaneous transmission of disturbances,
  which is incompatible with the assumption of a finite maximal propagation speed.

  Imposing bounded propagation requires that the effective dynamics suppresses large
  gradients in a smooth and intrinsic manner.
  Specifically, the Lagrangian density must satisfy the following conditions:
  (i) it reduces to a quadratic form in the small-gradient limit,
  (ii) it enforces a strict upper bound on $|\partial_\mu \phi|$,
  (iii) it introduces no additional microscopic length or energy scales beyond the
  bound itself,
  and (iv) it remains local and analytic in the admissible regime.

  Under these conditions, the functional form of $\mathcal{L}$ is highly constrained.
  Polynomial extensions of the quadratic action fail to enforce a strict bound without
  introducing additional scales or fine tuning.
  By contrast, a square-root structure provides a natural and self-regularizing mechanism
  that interpolates smoothly between the linear regime and a saturated regime at large
  gradients.

  The minimal Lagrangian density satisfying the above requirements is of Born--Infeld
  type,
  \begin{equation}
    \mathcal{L}_{\mathrm{BI}} =
    b^{2}
    \left(
      1 - \sqrt{1 - \frac{1}{b^{2}}\,
    \partial_\mu \phi\,\partial^\mu \phi}
    \right) ,
  \end{equation}
  where $b$ sets the maximal admissible magnitude of the gradient.
  In the limit $|\partial_\mu \phi| \ll b$, this expression reduces to the quadratic
  theory, while for large gradients it enforces strict saturation.

  The Born--Infeld structure is therefore not introduced as a phenomenological
  modification or a choice of convenience.
  It emerges as the unique effective representation, within the assumed class of local
  and scale-free functionals, that is compatible with bounded propagation.
  Configurations that would lead to singular behavior in a quadratic theory are smoothly
  regulated, and both the energy density and flux remain finite for all admissible field
  configurations.

  Importantly, the Born--Infeld functional does not define the microscopic dynamics of
  the underlying relational system.
  It provides an effective encoding of admissible continuum configurations in regimes
  where a local description is operationally meaningful.
  Outside these regimes, the continuum description itself ceases to apply, and no claim
  is made regarding the form of the dynamics beyond the saturation threshold.

  In the following sections, we show that this bounded-flux structure plays a central
  role in selecting the class of effective operators that admit a consistent geometric
  interpretation.
  In particular, it dynamically restricts the admissible continuum limits of relational
  Laplacians and thereby constrains the emergent spacetime geometries.
