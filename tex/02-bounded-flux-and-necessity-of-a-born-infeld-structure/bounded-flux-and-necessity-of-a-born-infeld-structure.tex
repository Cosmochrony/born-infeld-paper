\section{Bounded relaxation and effective action}
  \label{sec:bounded-relaxation-and-effective-action}

  We consider effective continuum descriptions arising from relational systems whose
  microscopic dynamics admits a finite maximal propagation or relaxation flux.
  Such a bound is a minimal physical requirement for causal consistency: no influence,
  constraint, or excitation can propagate arbitrarily fast in the effective theory.

  In a continuum regime, the dynamics of slowly varying configurations can be encoded
  in a local action functional
  \begin{equation}
    S = \int d^4x\,\mathcal{L},
  \end{equation}
  where $\mathcal{L}$ depends on the effective fields and their first derivatives.
  From the perspective of field theory, the central question is therefore the following:
  which local Lagrangian densities are compatible with bounded propagation?

  A purely quadratic action, such as the Maxwell form
  \begin{equation}
    \mathcal{L}_{\mathrm{quad}} \propto F_{\mu\nu}F^{\mu\nu},
  \end{equation}
  admits arbitrarily large field gradients.
  As a result, the associated fluxes and characteristic propagation scales are unbounded.
  This behavior is incompatible with the existence of a finite maximal relaxation or
  transport capacity at the microscopic level.

  Imposing bounded propagation requires the effective Lagrangian density to satisfy
  three minimal conditions:
  (i) it reduces to a quadratic form in the weak-field limit,
  (ii) it enforces a strict upper bound on admissible field invariants,
  and (iii) it introduces no additional microscopic scales beyond the saturation scale
  itself.
  Polynomial extensions of quadratic actions fail to meet these requirements without
  fine tuning or the introduction of auxiliary parameters.

  Under these assumptions, the effective Lagrangian density is uniquely constrained to
  take a Born--Infeld--type form.
  Anticipating the operator derivation developed in the following sections, the resulting
  effective action can be written as
  \begin{equation}
    S_{\mathrm{eff}}
    = \beta^2 \int d^4x
    \left(
      \sqrt{-\det\!\left(g_{\mu\nu} + \frac{1}{\beta} F_{\mu\nu}\right)}
      - \sqrt{-\det(g_{\mu\nu})}
    \right),
    \label{eq:BI-action}
  \end{equation}
  where $g_{\mu\nu}$ is the effective metric and $F_{\mu\nu}$ an antisymmetric field
  strength.
  The parameter $\beta$ sets the maximal admissible field magnitude and fixes the
  saturation scale of the theory.

  In the weak-field regime $|F_{\mu\nu}| \ll \beta$, the expansion of
  Eq.~\eqref{eq:BI-action} yields
  \begin{equation}
    \mathcal{L}_{\mathrm{eff}}
    = -\frac{1}{4}F_{\mu\nu}F^{\mu\nu}
    + \mathcal{O}\!\left(\beta^{-2}\right),\label{eq:equation-weak-field}
  \end{equation}
  showing that standard Maxwell electrodynamics is recovered universally as the
  low-flux approximation of bounded relaxation.
  Non-linear corrections become relevant only when the microscopic saturation scale
  is probed.

  Crucially, the Born--Infeld structure is not introduced here as a phenomenological
  modification of electrodynamics.
  It arises as the minimal local representation compatible with bounded propagation.
  This observation can be formulated as a no-go statement: any local effective action
  admitting a finite maximal flux cannot be purely quadratic in field gradients.
  Born--Infeld electrodynamics therefore represents the unique natural completion of
  Maxwell theory under bounded relaxation.

  In the present framework, the parameter $\beta$ is not free.
  It encodes the maximal transport or relaxation capacity of the underlying relational
  system and is fixed by its microscopic connectivity density and stiffness.
  In this sense, $\beta$ plays the role of an ultraviolet field scale, analogous to a
  Planck-scale bound, although no specific identification is assumed at this stage.

  The remainder of the paper establishes how the effective action
  \eqref{eq:BI-action} emerges dynamically from a bounded relational Laplacian and how
  this structure selects physically admissible spacetime geometries.
