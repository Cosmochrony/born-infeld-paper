\section{Relational graph, Laplacian operator, and relaxation dynamics}
  \label{sec:relational-graph-laplacian-operator-and-relaxation-dynamics}

  We consider an underlying relational system represented by a weighted graph
  $G = (V,E,w)$, where $V$ denotes a set of nodes, $E$ a set of edges, and
  $w_{ij} \ge 0$ encodes the strength of the relation between nodes $i$ and $j$.
  No embedding in a background spacetime is assumed.
  The graph is taken as a purely relational object, encoding adjacency and coupling
  structure between abstract degrees of freedom.

  A scalar field $\phi_i$ is defined on the nodes of the graph.
  The fundamental operator governing relational variations is the discrete Laplacian,
  defined by
  \begin{equation}
    (L \phi)_i = \sum_{j} w_{ij} \left( \phi_i - \phi_j \right) .
  \end{equation}
  This operator measures the local mismatch of $\phi$ with respect to its relational
  neighborhood and plays the role of a generalized stiffness or connectivity operator.
  Throughout this work, we assume that the graph is locally finite and that the weights
  are symmetric, $w_{ij} = w_{ji}$.

  The system is assumed to admit an irreversible relaxation dynamics driven by the
  Laplacian.
  At the discrete level, this dynamics can be represented schematically as
  \begin{equation}
    \frac{d \phi_i}{d \tau} = - (L \phi)_i ,
  \end{equation}
  where $\tau$ is a monotonically increasing parameter labeling the progression of the
  relaxation process.
  No interpretation of $\tau$ as a fundamental physical time is imposed.
  Instead, it serves as an ordering parameter associated with the irreversible flow
  toward admissible stationary configurations.

  This relaxation dynamics defines a preferred direction of evolution.
  Configurations evolve toward states that minimize relational gradients, subject to
  the constraints discussed in Section~2.
  The existence of such an ordering parameter is sufficient to distinguish one direction
  of evolution from its reverse, independently of any geometric notion of time.
  Temporal ordering is thus introduced operationally, as an intrinsic feature of the
  relaxation process itself.

  Stationary configurations satisfy
  \begin{equation}
    L \phi = 0 ,
  \end{equation}
  and correspond to relational equilibria.
  More generally, slowly varying configurations describe regimes in which the system
  admits an effective coarse-grained description.
  In these regimes, large-scale observables depend primarily on the spectral properties
  of the Laplacian rather than on the detailed microscopic structure of the graph.

  We emphasize that the graph structure introduced here does not imply any fundamental
  discreteness of physical space or time.
  It is employed as a relational scaffold allowing the definition of operators and
  relaxation processes.
  Different microscopic realizations leading to the same large-scale spectral properties
  are considered physically equivalent within the scope of the effective description.

  In the next section, we recall how, under appropriate density and regularity
  assumptions, the spectral structure of such relational Laplacians admits a continuum
  limit.
  In this limit, the discrete operator converges to a second-order differential operator,
  providing the bridge toward an effective geometric description.
