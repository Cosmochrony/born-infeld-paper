\section{Emergence of Minkowski spacetime from homogeneous relaxation}
  \label{sec:emergence-of-minkowski-spacetime-from-homogeneous-relaxation}

  We now consider the class of homogeneous and isotropic relaxation regimes in which the
  underlying relational system admits a stable and stationary coarse-grained description.
  Homogeneity is understood in a spectral sense: the low-energy sector of the relational
  Laplacian is invariant under translations generated by the relaxation dynamics, and no
  localized obstruction or defect is present.

  In such regimes, the effective continuum operator introduced in Section~4 simplifies
  considerably.
  The tensor $A^{\mu\nu}(x)$ becomes independent of position and reduces to a constant,
  symmetric matrix,
  \begin{equation}
    A^{\mu\nu}(x) = A^{\mu\nu}_0 .
  \end{equation}
  Isotropy further constrains its spatial components to be proportional to the identity.
  At this stage, no assumption is made regarding the signature of $A^{\mu\nu}_0$.

  A crucial ingredient is the existence of an irreversible relaxation dynamics, as
  introduced in Section~3.
  The relaxation parameter $\tau$ defines a preferred ordering of configurations, and
  thus singles out one distinguished direction associated with monotonic evolution.
  This direction cannot be treated on the same footing as the remaining directions,
  which describe relational diffusion within a given relaxation slice.
  As a result, the effective operator naturally decomposes into one ordering direction
  and a set of transverse directions.

  The bounded-flux condition established in Section~2 imposes a further restriction.
  Propagation along the ordering direction is constrained by a maximal admissible flux,
  while transverse propagation remains diffusive and isotropic.
  A purely Riemannian signature would place all directions on an equal footing and would
  allow arbitrarily fast propagation when combined with the relaxation ordering, in
  contradiction with the bounded-flux requirement.

  Consistency between irreversible ordering, isotropic diffusion, and bounded
  propagation therefore requires a pseudo-Riemannian structure with a single negative
  eigenvalue.
  Up to an overall scale, the unique admissible form of the effective metric is
  \begin{equation}
    g_{\mu\nu} = \mathrm{diag}(-1, +1, +1, +1) .
  \end{equation}
  This signature is not postulated but dynamically selected as the only one compatible
  with homogeneous relaxation and flux saturation.

  The resulting effective geometry is flat.
  All curvature invariants vanish identically, reflecting the absence of localized
  obstructions or inhomogeneities in the relational structure.
  The corresponding spacetime is therefore identified with Minkowski space, interpreted
  here not as a fundamental arena but as the effective description of a maximally
  symmetric and dynamically admissible relaxation regime.

  It is important to stress that this result does not rely on postulating Lorentz
  invariance at the microscopic level.
  Lorentz symmetry emerges as a property of the homogeneous fixed point selected by the
  dynamics.
  Departures from homogeneity or from the bounded-flux regime lead to deviations from
  flat geometry, as discussed in the following sections.

  In summary, homogeneous relaxation of a relational system subject to bounded
  propagation uniquely selects a flat effective spacetime with signature $(- + + +)$.
  This provides a dynamical explanation for the emergence of Minkowski geometry without
  introducing independent metric degrees of freedom or relativistic postulates.
