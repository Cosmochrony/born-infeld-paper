\section{Emergence of Minkowski Spacetime: Spectral Stability and Signature Selection}
  \label{sec:emergence-of-minkowski-spacetime}

  We now demonstrate that the class of homogeneous and isotropic relaxation regimes uniquely selects a flat
  pseudo-Riemannian metric.
  In this framework, homogeneity is defined spectrally: the low-energy sector of the relational Laplacian $\mathcal{L}$
  is invariant under the translation group, implying that the effective tensor $A^{\mu\nu}(x)$ reduces to a constant
  matrix $A^{\mu\nu}_0$.

  The selection of the signature follows from the requirement of \textit{causal consistency}
  between the irreversible relaxation dynamics and the bounded-flux condition.
  Let $\tau$ be the relaxation parameter defining the monotonic evolution of the relational system.
  The effective propagator associated with the bounded-flux operator must satisfy a well-posed initial value problem
  relative to $\tau$.

  Consider the effective dispersion relation derived from the saturated regime of Eq.~\eqref{eq:equation-weak-field}.
  In a local coordinate basis where the ordering direction is $\partial_\tau$, the bounded propagation requirement
  translates into a constraint on the characteristic cone of the operator.
  If we denote $k_0$ as the frequency conjugate to $\tau$ and $\mathbf{k}$ the spatial momenta, the stability of the
  relaxation process requires that the operator remains hyperbolic~\cite{HormanderPDE},
  ensuring a well-posed Cauchy problem.
  A purely Riemannian signature $(++++)$ would lead to an elliptic operator, for which no finite propagation cone
  exists and perturbations propagate instantaneously across the entire relational structure,
  violating the bounded-flux postulate established in
  Section~\ref{sec:bounded-relaxation-and-effective-action}.

  Conversely, a signature with multiple time-like directions would lead to ultra-hyperbolic equations, which are
  generically unstable and fail to support a well-defined causal structure or a well-posed initial value
  problem~\cite{ChoquetBruhatGR}.
  Consequently, the only spectrally stable configuration that allows for a well-posed, monotonic relaxation
  under a finite maximal flux is the one where the ordering direction $\tau$ possesses a sign opposite to the diffusive
  spatial directions.
  By normalizing the maximal relaxation speed to $c=1$, the effective metric is dynamically fixed to:
  \begin{equation}
    g_{\mu\nu} = \eta_{\mu\nu} = \mathrm{diag}(-1, +1, +1, +1).
  \end{equation}

  This result implies that Lorentz invariance is an \textit{emergent symmetry}
  of the homogeneous fixed point of the relaxation dynamics.
  The Minkowski metric is not a background stage but the
  unique effective representation of a maximally symmetric relational state at equilibrium.

  Crucially, the emergence of the $(-+++)$
  signature provides a physical justification for the use of Wick rotations in standard field theory:
  here, the ``Euclidean'' sector describes the relational connectivity at a fixed relaxation slice,
  while the Lorentzian signature encodes the dynamical unfolding of these relations.
  In this light, the null interval $ds^2=0$ corresponds to the saturation of the relational flux,
  where information transfer reaches the bound imposed by the microscopic connectivity of $\chi$,
  a mechanism closely analogous to the causal saturation encountered in non-linear
  Born--Infeld-type field theories~\cite{GibbonsBornInfeldCausality}.

  In the absence of localized obstructions, all curvature invariants $R_{\mu\nu\rho\sigma}$
  vanish identically, and the geometry is flat.
  We shall see in the next section how localized defects break this symmetry and generate the
  Schwarzschild effective geometry through the same saturation mechanism.
